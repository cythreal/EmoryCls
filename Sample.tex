% ------------------------- emory.cls Sample Document --------------------------
% This is a sample document for emory.cls 
% This document includes the contents likely to be included in a dissertation,
% such as university required cover pages, table of contents, list of tables
% and figures, chapters, floats (figures, tables, math equations, etc.), 
% citations, footnotes, appendices, and indices. 
% Compile this document using the emory.cls class file should sucessfully 
% generate a .pdf file with the desired dissertation format.
%
% The dissertation style instructions can be found at
% ------ documentclass declaration ---------------------------------------------
% --- \documentclass[options]{emory}
% --- Only [draft] and [final] are allowed.
% --- Other properties can be set by keys.
\documentclass[final]{emory}

% --- Set up other document properties. Available keys and values:
% --- option: document options
%     --  dissertation=<boolean>
%           true: document is a dissertation [default]
%           false: document is a thesis
%     --  twoside=<boolean>
%           true: twoside printing. 
%                 Alternates left and right margins on even/odd pages
%           false: oneside printing. Uniform margin widths.
% --- preset: preset styles

\setkeys[emory]{option}{twoside=false}
\setkeys[emory]{option}{dissertation=true}
\setkeys[emory]{preset}{chapter=true}
\setkeys[emory]{preset}{fig=true}
\setkeys[emory]{preset}{list=true}
\setkeys[emory]{preset}{table=true}
\setkeys[emory]{preset}{toc=true}

% ------ set up bibliography style ---------------------------------------------
% --- Preset style, driven by BibLaTeX, are available for each major.
% --- need to use \autocite instead of \cite
% --- use command \presetbib{bibsource.bib} to set up the bibliography file
\presetbib{SampleBib.bib}
% --- Or, you can load your own bibtex packages and set up your own styles.


% ------ include your own packages ---------------------------------------------
\usepackage{url}
\usepackage{fancyvrb}
\definecolor{verbcolor}{rgb}{0.1,0.15,0.65}
\fvset{%
  fontsize=\small,
	formatcom=\color{verbcolor},
	baselinestretch=1.0,
	numbers=left,
	numbersep=1ex,
	gobble=-2,
	xleftmargin=4ex
}

\usepackage[colorlinks=true]{hyperref}
% This line of code is to remove compatibility issue with hyperref
\AtBeginDocument{\let\textlabel\label}

% ------ title and author and the boss -----------------------------------------
\title{A \LaTeX{} Class for the Dissertation Style of Emory University}
\author{Luyao Zou}                      % replace with your legal name
\degree{Doctor of Philosophy}           % replace with your degree, if necessary
\prevdegree{B.S., Fudan University, 2012}        % replace with your previous degree, repeat if necessary
\field{Your field}
% \advisor{Name}{Degree}, same for committee.
% Repeat command for multiple inputs. Please manually sort with alphabatical order
\advisor{Your dear advisor}{Ph.D}
\advisor{Your co-advisor}{Ph.D}
\committee{Dr.\ Alfalfa}{Ph.D}
\committee{Dr.\ Banana}{M.D}
\committee{Dr.\ Cashew}{M.D \& Ph.D}
\committee{Dr.\ Donut}{J.D}
\committee{Mr.\ Eggplant}{M.S.}
\yr{2016}                               % replace the year, if necessary
\dean{Lisa A.\ Tedesco, Ph.D}     % The dean. Leave it as is


% ------ start your awesome writing here ---------------------------------------
\begin{document}

% ------ abstract --------------------------------------------------------------
\begin{abstract}

This is a \LaTeX{} class for the dissertation/thesis style of Laney Graduate School (LGS) of Emory University.
It is designed to meet the minimum dissertation format
requirement of LGS, but nothing is guaranteed. Please tweak the
code when necessary. This class is licensed with GNU Generic Public License (GPL) v3.0.

\end{abstract}

% ------ acknowledgement -------------------------------------------------------
\begin{acknowledgement}%

I would like to thank all the documentations and anonymous people online that
helped me learn the \LaTeX{} skills to write this class. 

A list of references and code examples I consulted during the development
of this class:
\begin{itemize}
  \item The American Chemical Society journal package \Verb|achemso|:
  
        \url{https://www.ctan.org/pkg/achemso?lang=en}
  \item \LaTeXe{} for class and package writers:
        
        \url{https://www.ctan.org/pkg/clsguide}
  \item Peter Flynn (2007), ``Rolling your own Document Class: Using \LaTeX{} to keep away from the Dark Side'', Proceedings of the Practical \TeX{} 2006 Conference, TUGboad, Vol 28.
        
        \url{https://tug.org/pracjourn/2006-4/flynn/}
  \item Jim Hefferon (2005), ``Minutes in Less Than Hours: Using \LaTeX{} Resources'', TPJ No.~04: 
        \url{https://tug.org/pracjourn/2005-4/hefferon/} 
  \item The thesis class originally written by Bruce Fast, and most recently updated by Hongcheng Ni, from the University of Colorado, Boulder. 
        
        \url{http://www.colorado.edu/oit/software-hardware/tex-latex/thesis-class/documentation-sample-files}
  \item \Verb|ifthen| package: \url{http://texdoc.net/texmf-dist/doc/latex/base/ifthen}
  \item \Verb|geometry| package: \url{https://www.ctan.org/pkg/geometry?lang=en}
  \item \Verb|xkeyval| package: \url{https://www.ctan.org/pkg/xkeyval?lang=en}
  \item \Verb|biblatex| package: \url{https://www.ctan.org/pkg/biblatex?lang=en}
\end{itemize}

Special thanks to Dr.~Ulf Nilsson at the Laney Graduate School of Emory University
for the general advice for the style.

\end{acknowledgement}

\maketoc

\chapter{Introduction}
%Putting up a dissertation following all the requirements or restrictions of
%your university policy may drive you nuts. This is why \LaTeX{} is glorious. 
%Unfortunately, Emory does not provide an ``official'' \LaTeX{} style package for
%her brilliant, hard-working Ph.D.\ candidates. I here try my best to offer a
%third-party \LaTeX{} class that hopefully covers most of the Emory policy, if not
%all.

This is a \LaTeX{} class designed for dissertation writing under the style and
format policy of \href{http://www.graduateschool.emory.edu/uploads/academics/completion/Submit%20Thesis%20or%20Dissertation.doc}{the Laney Graduate School at Emory University} \autocite{emory-style}.
This class is
intended to reduce the effort of formating for Emory graduate students, thus 
allowing them to focus more on the writing. 

To report bugs and suggestions, please contact me on GitHub. To contribute, 
please fork my github repo at \url{https://github.com/luyaozou/EmoryCls.git}

\chapter{Usage}
\section{Load the class and compilation}
The Emory class can be loaded by declaring
\Verb|\documentclass{emory}| in the \LaTeX{} preamble. The class file
\Verb|emory.cls| needs to be in the same directory of the \LaTeX{} file(s) to be 
compiled. 

Alternatively, you may copy \Verb|emory.cls| into your local \TeX{} package 
directory, similar to \Verb|C:\MiKTeX\tex\latex\emory\| in Windows systems, or\\
\Verb|/usr/share/texmf-texlive/tex/latex/emory/| in Mac and Linux systems. After
creating an \Verb|emory| folder and putting the class file in it, run
\Verb|texhash| to update the \TeX{} package registry. Remember, you may need
root permission to complete these operations. 

Use \Verb|pdflatex| instead of \Verb|latex| as the compilation driver. This
is because certain packages, such as \Verb|url|, rely on pdf drivers.

The preset bibliography manager is \Verb|biblatex| with \Verb|biber| back-end. 
Users are encouraged to set-up their own \BibTeX{} driver for customized bibliography styles.


\section{Set document options}\label{sec:options}
The options of this class can be set in the \LaTeX{} preamble. They are 
self-explanatory in the \Verb|Sample.tex| file.
Compiling \Verb|Sample.tex| with the \Verb|pdflatex| driver yields a pdf version of this documentation.

\subsection{Document style options}
The first set of options in the \Verb|Emory| class is the \Verb|\documentclass| option.
Only \Verb|[draft]| and \Verb|[final]| are allowed in the 
\Verb|\documentclass[options]{emory}| command.
\Verb|[draft]| only draws the boarder of graphs without actually loading the images,
and prints solid black rectangles on any places that are outside the defined text area. 
This mode is useful for checking any element that exceeds the boarder,
which is fatal to the submission of the thesis/dissertation.
\Verb|[final]| produces the final document. 

Following the \Verb|\documentclass| command are several keys--value pairs.\\
Use \Verb|setkeys[emory]{family}{key=value}| to assign values.

\Verb|\setkeys[emory]{option}{boolkey=true/false}| configures the general document options.
Two options are available: \Verb|dissertation| and \Verb|twoside|.

\Verb|dissertation=true/false| configures the type of the document. 
It only does one thing: print out either ``dissertation'' or ``thesis'' on the 
Approval Pages.

\Verb|twoside=true/false| controls the double side printing, which alternates the left and right margins on odd/even pages for correct binding.
\textbf{The default value is \Verb|twoside=false| as it is the preference of the Laney Graduate School for electronic thesis and dissertation (ETD)}. 
Please note that the special pages are \emph{ALWAYS} single side printing. 
Therefore, if you choose \Verb|twoside=true|, \emph{remember to print out the special pages separately}.

\Verb|\setkeys[emory]{preset}{boolkey=true/false}| controls the switches for preset styles.
For users' convenience, a set of general dissertation styles has been configured in the class. 
This style set includes configurations for the styles of chapter and section 
titles, the figure environment, the table environment, the list environment, 
and the table of contents. Use boolkey \Verb|chapter|, \Verb|fig|, \Verb|table|, 
\Verb|list|, and \Verb|toc| to access these options.

If you want to turn keys back on after they have been turned off, you may need 
to recompile the document more than once to get the correct output.

\subsection{{\BibTeX{}} treatment}
Setting up the \BibTeX{} style is perhaps the most complicated task in \LaTeX{} 
because there are countless ways of defining citation styles, which vary from 
discipline to discipline and are highly personal.
To minimize the size of this \LaTeX{} class, only one preset \BibTeX{} style
is offered. It can be specified by \Verb|\presetbib{bibtex-source.bib}|.
This style is a numeric citation style with bibliography items sorted by their citation order,
and the citation is wrapped by square brackets. 
This style is driven by the \Verb|biblatex| package with \Verb|biber| as the 
back-end. 
The use of \Verb|biber| is a bit different from the legacy \Verb|bibtex| driver.
One needs to run \Verb|biber Sample.bcf| to generate the correct \Verb|.aux| 
file that will then be written into the \Verb|.pdf| in the next run of 
\Verb|pdflatex| compilation. For more information of \Verb|biblatex|, check its 
package page on the CTAN archive. 

If citation styles other than the preset style is desired, the user is 
encouraged to configure their own \BibTeX{} driver and style.
It is because the countless ways of doing so that I decided not to 
over-complicate the class preset.
No \BibTeX{} package or configuration will be loaded in the class if the 
command \Verb|\presetbib{}| is not specified. 
Therefore, the class file should be compatible with all common \BibTeX{} 
drivers and packages, such as \Verb|natbib|. 


\subsection{Special page entries}
The second set of options in the preamble configure the information needed for the special pages:

\begin{itemize}
\item \Verb|\title{Title}| sets up the title of the dissertation/thesis. The title 
will be reprinted several times in the special pages.
\item \Verb|\author{Name}| sets up the name of the author. The name will be reprinted
several times in the special pages. 
\item \Verb|\degree{Ph.D./Master}| sets up the degree you apply. ``It is either Doctor
of Philosophy or Master of Arts/Science/or another area.  There are a number of 
possible Master's degree in some programs, particularly Music and Education.  
If you are unsure, consult with your Director of Graduate Studies.'' \autocite{emory-style}
\item \Verb|\prevdegree{B.S.}| sets up your previous academic degree. ``List your 
degree, the name of the university, and the year you received the degree.  
For example, `B.A., Yale University, 2004' or `M.Sc., University of Pennsylvania, 2003'.''
You can list several degrees, including master's degrees from your current 
program here at Emory, by repeating this command.
``Note that you can only list a master's degree from your
Emory program if you actually applied for and received it; even if master's 
degrees are given on the basis of candidacy, they are not automatic.'' \autocite{emory-style} 
\item \Verb|\field{Field}| sets up your field of study. ``Generally, 
your field is the subject-matter name of your program: English, Chemistry, 
Health Services Research and Health Policy, Educational Studies, Business, etc.  
The field does not include the word `program,' and does not include any sub-field
you specialize in.  For example, your field is not `19th Century American Literature,'
`Finance Program,' `Neurology Department,' `African Studies' or any other
sub-field or entity.''\autocite{emory-style}
\item \Verb|\advisor{Advisor Name}{Advisor Degree}| sets up the name and degree of your 
advisor. ``If your committee had co-chairs, refer to each of them as Advisor.'' \autocite{emory-style} 
This can be done by repeating this command, and list the names of your advisors in alphabetical order.
\item \Verb|\committee{Committee Member Name}{Committee Member Degree}| sets 
up\\
the name and degree of your committee member. Repeat this command for multiple 
members. If the number of committee members exceeds 4, the class will change
the signing sheet into a two column style instead of one column style.
\item \Verb|\yr{2016}| sets up the calendar year you receive your degree
(not the academic year).
\item \Verb|\dean{}| sets up the name of the current Dean of the LGS. There is no current need to modify this entry.
\end{itemize}


\section{Recommended packages}
The following \LaTeX{} packages are automatically pre-loaded by the class: \Verb|amsmath|, 
\Verb|amssymb|, \Verb|color|, \Verb|mhchem|, \Verb|graphicx|, \Verb|titlesec|, 
\Verb|titletoc|, \Verb|ifthen|, \Verb|xkeyval|, \Verb|geometry|, and 
\Verb|fancyhdr|. All the packages listed here are standard \LaTeX{} packages 
that should be shipped with any standard \LaTeX{} distribution. If you cannot 
find a certain package in your local computer system, check out the CTAN 
archive \url{https://www.ctan.org/}.

More external packages can be loaded via \Verb|\usepackage[options]{package}|
in the preamble. Particularly, if you want to use \Verb|hyperref| to create hyperlinks, 
you need to also add \Verb|\AtBeginDocument{\let\textlabel\label}| in the preamble to fix the compatibility issue. 


\section{Use environments}\label{sec:env}
The Emory class should be fully compatible with standard \LaTeX{} environments:
\Verb|figure|, \Verb|table|, \Verb|quote|, \Verb|itemize|, \Verb|enumerate|,
and \Verb|verbatim|.
%In addition to these standard enviroments, \Verb|emory.cls| also defines two new
%enviroments: \Verb|prose| and \Verb|poetry|, for bulk quotation of different literature genre.

Sample outputs of these environments are listed here. 

\subsection{Figures}

\begin{figure}[htp!]
  \centering
  \includegraphics[width=0.5\textwidth]{LaTeXLogo.png}
  \caption[This is a figure environment]{This is a figure 
  environment.\label{fig} If the \protect\Verb|preset| option 
  for \protect\Verb|fig| is turned on, the spacing between the graphics and the 
  figure caption is adjusted as printed in this example for better visual 
  effect.}
\end{figure}

\subsection{Schematics}

\Verb|schematic| is a customized environment inherited from the \Verb|figure|
environment. It changes the caption head from ``Figure'' to ``Schematic''. Do 
not include float position option
\Verb|[htp!]|, as it is already set up in the environment. The schematic caption will 
remain in the list of figures, and will have numbering 
consistent with the \Verb|figure| floats because it is an environment inherited from the \Verb|figure| environment.

\begin{schematic}
	\centering
  \includegraphics[width=0.5\textwidth]{LaTeXLogo.png}
  \caption{This is not a actually schematic.\label{schematic}}
\end{schematic}


\subsection{Tables}

\begin{table}[htp!]
  \centering
  \caption[This is a simple table]{This is a simple table. \label{table} If the 
  \protect\Verb|preset| 
  option 
  	for \protect\Verb|table| is turned on, the spacing between the table and 
  	the 
  	caption is adjusted as printed in this example for better visual effect. 
  	The width of the table line (0.5~pt) and the text line spacing (single 
  	spacing) are also configured.}
  \begin{tabular}{l p{0.6\textwidth}}
    \hline
    Sample & Table \\
    \hline 
    row & 1 \\
    row & 2 \\
    row & 3 \\
    row & 4 \\
    \hline
  \end{tabular}
\end{table}

\subsection{Mathematical equations}
Single mathematical equation (needs the \Verb|amsmath| package)
\begin{equation}
  E = mc^2 \label{math:single}
\end{equation}

Multiple mathematical subequations (need the \Verb|amsmath| package)
\begin{subequations}\label{math:sub}
\begin{align}
  \nabla\cdot\pmb{E}&=\frac{\rho}{\epsilon_0} \label{math:sub1}\\
  \nabla\cdot\pmb{B}&=0 \label{math:sub2} \\
  \nabla\times\pmb{E}&=-\frac{\partial\pmb{B}}{\partial t} \label{math:sub3}\\
  \nabla\times\pmb{B}&=\mu_0\Big(\pmb{J}+\epsilon_0\frac{\partial\pmb{E}}{\partial t}\Big) \label{math:sub4}
\end{align}
\end{subequations}

\subsection{Block quotes}
Quote environment. Use \Verb|\begin{quote} \end{quote}|

\begin{quote}\label{poet}
O wild West Wind, thou breath of Autumn's being,

Thou, from whose unseen presence the leaves dead

Are driven, like ghosts from an enchanter fleeing,
\end{quote}

\subsection{Lists and enumerates}
List and enumerate environment. Use \Verb|\begin{itemize} \item \end{itemize}|
and \Verb|\begin{enumerate} \item \end{enumerate}|. List and enumerate 
environments
can be nested.

A nested list structure:
\begin{itemize}
  \item This is a list item.
  \begin{itemize}
    \item This is a nested list item.
    \begin{itemize}
      \item This is a nested nested list item.
      \begin{itemize}
        \item This is a nested nested nested list item. 
        \item The bullets for 4 levels are defined in \Verb|emory.cls|.
        \item I would strongly doubt you need more than 4 levels of nested lists.
      \end{itemize}
    \end{itemize}
  \end{itemize}    
\end{itemize}

A nested enumerate structure:
\begin{enumerate}
  \item This is a enumerate item.
  \begin{enumerate}
    \item This is a nested enumerate item.
    \begin{enumerate}
      \item This is a nested nested enumerate item.
      \begin{enumerate}
        \item This is a nested nested nested enumerate item. 
        \item The numbering for 4 levels are defined in \Verb|emory.cls|.
        \item I would strongly doubt you need more than 4 levels of nested lists.
      \end{enumerate}
    \end{enumerate}
  \end{enumerate}    
\end{enumerate}

\section{Verbatim}

Use \Verb|\verb?<code>?| to typeset inline code  (\Verb|?| can be replaced by
other paired special characters),
or  \Verb|\begin{verbatim} <code> \end{verbatim}| to typeset code blocks.
The package \Verb|fancyvrb| offers inline \Verb|\Verb??| and \Verb|Verbatim| environment with more enhanced
options (note that the ``V'' is capitalized). The customized verbatim environment is single spaced, and may
apply a smaller font to fit the page size limit. This is acceptable.
Please note that any codes, including frames and line numbers, shall not
exceed the default margin. Therefore you need to tweak the indentation of your code blocks if necessary.
Note that the \Verb|draft| mode does not print warning in \Verb|fancyvrb| environments.

\begin{Verbatim}
% The code to set up fancyvrb style shown in this sample:
\usepackage{fancyvrb}
\definecolor{verbcolor}{rgb}{0.1,0.15,0.65}
\fvset{%
  % set font size
  fontsize=\small
  % set font color
  formatcom=\color{verbcolor},	
  % use single spacing
  baselinestretch=1.0,
  % set numbering of lines				  
  numbers=left,									
  % set the spacing between line number and code
  numbersep=1ex,
  % set number of characters to suppress at the beginning of each line
  gobble=-2,
  % set left margin
  xleftmargin=4ex
}
\end{Verbatim}


\section{Cross reference}
One advantage of \LaTeX{}, out of many, is the convenience of cross referencing in
the text. Use \Verb|\label{yourlabel}| to label the material, and \Verb|\ref{yourlabel}|
to reference it in some other places in your text. And you do not need to worry
about numbering: \LaTeX{} does it automatically for you. For example, here
I will reference the sample environments in the previous section (Section 
\ref{sec:env}):
\begin{itemize}
\item The label for Table~\ref{table} is \Verb|\label{table}|.
\item The label for Figure~\ref{fig} is \Verb|\label{fig}|.
\item The label for Equation~\ref{math:single} is \Verb|\label{math:single}|
\item The label for Equations~\ref{math:sub} is \Verb|\label{math:sub}|. And then
      you can reference to each subequation: \ref{math:sub1}, \ref{math:sub2}, 
      \ref{math:sub3}, and \ref{math:sub4} via \Verb|\label{math:sub1}|,
      \Verb|\label{math:sub2}|, \Verb|\label{math:sub3}|, and \Verb|\label{math:sub4}|,
      respectively.
\item The label for Quote~\ref{poet} is \Verb|\label{poet}|.
\end{itemize}


\appendix
% change chapter name to 'Appendix'
\renewcommand{\chaptername}{Appendix}

\chapter{Appendix}
This is an appendix. The numbering of appendix changes from Arabic numbering to 
alphabetical. Use \Verb|\renewcommand{\chaptername}{Appendix}| to switch
the name ``Chapter'' to ``Appendix'' before the numbering.
The chapter and section titles will still print in the table of contents.

\section{Appendix section}

\end{document}
