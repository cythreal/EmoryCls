% ------------------------- emory.cls Sample Document --------------------------
% This is a sample document for emory.cls 
% This document includes the contents likely to be included in a dissertation,
% such as university required cover pages, table of contents, list of tables
% and figures, chapters, floats (figures, tables, math equations, etc.), 
% citations, footnotes, appendices, and indices. 
% Compile this document using the emory.cls class file should sucessfully 
% generate a .pdf file with the desired dissertation format.
%
% The dissertation style instructions can be found at
% ------ documentclass declaration ---------------------------------------------
% --- \documentclass[options]{emory}
% --- Only [draft] and [final] are allowed.
% --- Other properties can be set by keys.
\documentclass[draft]{emory}

% --- Set up other document properties. Available keys and values:
% --- style: select style by major=<major>
%            supported majors: chem, phys, math, bio, and cs 
%            select printing by twoside=<boolean>
%            true: alternating left/right margins for odd/even pages.
%            false: consistent margins throughout the document.
% --- doc: dissertation=<boolean>
%          true: document is a dissertation [default]
%          false: document is a thesis
%          
\typeout{I'm loading keys}
\setkeys[emory]{style}{major=chem}
\setkeys[emory]{style}{twoside=false}
\setkeys[emory]{doc}{dissertation=true}

% ------ set up bibliography style ---------------------------------------------
% --- Preset style, driven by BibLaTeX, are available for each major.
% --- need to use \autocite instead of \cite
% --- use command \presetbib{bibsource.bib} to set up the bibliography file
\presetbib{SampleBib.bib}
% --- Or, you can load your own bibtex packages and set up your own styles.


% ------ include your own packages ---------------------------------------------
\usepackage{url}
\usepackage{fancyvrb}
\definecolor{verbcolor}{rgb}{0.1,0.15,0.65}
\fvset{formatcom=\color{verbcolor}}

% ------ title and author and the boss -----------------------------------------
\title{A \LaTeX{} Class for the Dissertation Style of Emory University}
\author{Luyao Zou}                      % replace with your legal name
\degree{Doctor of Philosophy}           % replace with your degree, if necessary
\prevdegree{B.S., Fudan University, 2012}        % replace with your previous degree, repeat if necessary
\field{Your field of expertise}
% \advisor{Name}{Degree}, same for committee.
% Repeat command for multiple inputs. Please sort with alphabatical order
\advisor{Your dear advisor}{Ph.D}
\advisor{Your co-advisor}{Ph.D}
\committee{Dr.\ Alfalfa}{Ph.D}
\committee{Dr.\ Banana}{M.D}
\committee{Dr.\ Cashew}{M.D \& Ph.D}
\committee{Dr.\ Donut}{J.D}
\committee{Mr.\ Eggplant}{M.S.}
\yr{2016}                               % replace the year, if necessary


% ------ start your awesome writing here ---------------------------------------
\begin{document}

% ------ abstract --------------------------------------------------------------
\begin{abstract}

This is a \LaTeX{} class. It is supposed to follow the dissertation format
requirement of Emory University, but nothing is guaranteed, and please tweak the
code when necessary. 

\end{abstract}

% ------ acknowledgement -------------------------------------------------------
\begin{acknowledgement}%

I would like to thank all the documentations and anonymous people online that
helped me learn the \LaTeX{} skills to write this class. 

An imcomplete list of references and code examples I consulted during the development
of this class:
\begin{itemize}
  \item The Amecican Chemical Society journal package \Verb|achemso|:
  
        \url{https://www.ctan.org/pkg/achemso?lang=en}
  \item \LaTeXe{} for class and package writers:
        
        \url{https://www.ctan.org/pkg/clsguide}
  \item Peter Flynn (2007), ``Rolling your own Document Class: Using \LaTeX{} to keep away from the Dark Side'', Proceedings of the Practical \TeX{} 2006 Conference, TUGboad, Vol 28.
        
        \url{https://tug.org/pracjourn/2006-4/flynn/}
  \item Jim Hefferon (2005), ``Minutes in Less Than Hours: Using \LaTeX{} Resources'', TPJ No.~04: 
        \url{https://tug.org/pracjourn/2005-4/hefferon/} 
  \item Oren Patashnik (1988), ``Designing \BibTeX{} Styles'':
        \url{http://www.pctex.com/files/managed/a/a3/btxhak.pdf}
  \item The thesis class originally written by Bruce Fast, and most recently updated by Hongcheng Ni, from University of Colorado, Boulder. 
        
        \url{http://www.colorado.edu/oit/software-hardware/tex-latex/thesis-class/documentation-sample-files}
  \item Kate L.\ Turabian, \textit{A Manual for Writers of Research Papers, Theses, and 
  Dissertations}, 8\textsuperscript{th}ed, ed.\ Wayne C.\ Booth, Gregory G.\ Colomb, Joseph M.\ Williams, and the University of Chicago Press Editorial Staff,  Chicago IL: University of Chicago Press, 2007.
  \item \Verb|ifthen| package: \url{http://texdoc.net/texmf-dist/doc/latex/base/ifthen}
  \item \Verb|geometry| package: \url{https://www.ctan.org/pkg/geometry?lang=en}
  \item \Verb|xkeyval| package: \url{https://www.ctan.org/pkg/xkeyval?lang=en}
\end{itemize}

Special thank to Dr.~Ulf Nilsson at the Laney Graduate School of Emory University
for the general advice of the style.

\end{acknowledgement}

\maketoc

\chapter{Introduction}
%Putting up a dissertation following all the requirements or restrictions of
%your university policy may drive you nuts. This is why \LaTeX{} is glorious. 
%Unfortunately, Emory does not provide an ``official'' \LaTeX{} style package for
%her brilliant, hard-working Ph.D.\ candidates. I here try my best to offer a
%third-party \LaTeX{} class that hopefully covers most of the Emory policy, if not
%all.

This is a \LaTeX{} class designed for dissertation writing under the style and
format policy of the Laney Graduate School at Emory University.\autocite{emory-style}
This class is
intended to reduce the effort in formating for Emory graduate students, thus 
to focus more on the writing. 

To report bugs and suggestions, please contact me on GitHub. To contribute, 
please fork my github repo at \url{https://github.com/luyaozou/EmoryCls.git}

\chapter{Usage}
\section{Load the class and compilation}
The Emory class can be loaded by declaring
\Verb|\documentclass{emory}| in the \LaTeX{} preamble. The class file
\Verb|emory.cls| needs to be in the same directory of the \LaTeX{} file(s) to be 
compiled. 

Alternatively, you may copy \Verb|emory.cls| into your local \TeX{} package 
directory, similar to \Verb|C:\MiKTeX\tex\latex\emory\| in Windows systems, or\\
\Verb|/usr/share/texmf-texlive/tex/latex/emory/| in Mac and Linux systems. After
creating an \Verb|emory| folder and putting the class file in it, run
\Verb|texhash| to update the \TeX{} package registery. Remember, you may need
root permission to complete these operations. 

Use \Verb|pdflatex| instead of \Verb|latex| as the compilation driver. This
is because certain packages, such as \Verb|url|, relies on pdf driver. 


\section{Set document options}\label{sec:options}
The options of this class can be set in the \LaTeX{} preamble. They are 
self-explanotary in the \Verb|sample.tex| file. 

\subsection{Document style options}
The first set of options is the document class option and keys.
Only \Verb|[draft]| and \Verb|[final]| are allowed in the \Verb|\documentclass[options]{emory}|command.
\Verb|[draft]| only draws the boarder of graphs without actually loading the images,
and prints solid black rectangles on any places that are outside the defined text area. 
This mode is useful for checking any element that exceeds the boarder,
which is fatal to the submission of the thesis/dissertation.
\Verb|[final]| produces the final document. 

Several keys--value pairs are available right after \Verb|\documentclass| statement.
Use \Verb|setkeys[emory]{family}{key=value}| to assign values.

\Verb|\setkeys[emory]{style}{major=}| configures the text, paragraph, and 
bibliography style based on the conventions
from specfic majors, and load up commonly used packages for that major.
The majors can be set by \Verb|major=value|.
Currently, chemistry style \Verb|chem| is supported. More major specific styles
settings will be added in the future after sufficient commminication with 
the scholars in these particular specialties. 

\Verb|\setkeys[emory]{style}{twoside=true/false}| controlls the double side printing.
The default value is \Verb|twoside=false| as the preference of the Laney Graduate School for easily read electronic files. 
However, if you prefer double side printing for your paper copy and do no want 
to mess up your binding,
you can set this key value to \Verb|true| so that the left and right margins will alternate for odd and even pages. 

Please note that the special pages are \emph{ALWAYS} single side printing. 
Therefore, if you have a double side printing document, \emph{remember to print out
the special pages separately}.


\Verb|\setkeys[emory]{doc}{dissertation=true/false}| configures the type of the 
document. It only does one thing: print out either ``dissertation'' or ``theis''
in the Approval Pages.

The \Verb|emory| class relies on several standard \LaTeX{} packages. Meanwhile, 
packages commonly used in special areas are also loaded when selecting the 
corresponding \Verb|major|. 
Table~\ref{tbl:packages} lists packages loaded by \Verb|emory.cls|.
\begin{table}[htp!]
  \centering
  \caption{Packages loaded by \protect\Verb|emory.cls|. \label{tbl:packages}}
  \begin{tabular}{l l}
    \hline
    Major & Package \\
    \hline
    Shared & \protect\Verb|ifthen|, \protect\Verb|xkeyval|, \protect\Verb|amsmath|,
             \protect\Verb|graphix|, \protect\Verb|titlesec|, 
             \protect\Verb|titletoc|, \protect\Verb|color| \\
    math & \protect\Verb|amsfonts|, \protect\Verb|amssymb| \\
    phys & \protect\Verb|mhchem| \\
    chem & \protect\Verb|mhchem| \\
    bio & \\
    cs & \protect\Verb|fancyvrb| \\
    \hline
  \end{tabular}
\end{table}


\subsection{Special page entries}
The second set of preambles configures the information need for the special pages:

\begin{itemize}
\item \Verb|\title{Title}| sets up the title of the dissertation/thesis. The title 
will be reprinted several times in the special pages.
\item \Verb|\author{Name}| sets up the name of the author. The name will be reprinted
several times in the special pages. 
\item \Verb|\degree{Ph.D./Master}| sets up the degree you apply. ``It is either Doctor
of Philosophy or Master of Arts/Science/or another area.  There are a number of 
possible Master's degree in some programs, particularly Music and Education.  
If you are unsure, consult with your Director of Graduate Studies.''\autocite{emory-style}
\item \Verb|\prevdegree{B.S.}| sets up your previous academic degree. ``List your 
degree, the name of the university, and the year you received the degree.  
For example, `B.A., Yale University, 2004' or `M.Sc., University of Pennsylvania, 2003'.
You can list several degrees, including master's degrees from your current 
program here at Emory, by repeating this command.
Note that you can only list a master's degree from your
Emory program if you actually applied for and received it; even if master's 
degrees are given on the basis of candidacy, they are not automatic.''\autocite{emory-style} 
\item \Verb|\field{Field}| sets up your field of study. ``Generally, 
your field is the subject-matter name of your program: English, Chemistry, 
Health Services Research and Health Policy, Educational Studies, Business, etc.  
The field does not include the word `program,' and does not include any sub-field
you specialize in.  For example, your field is not `19th Century American Literature,'
`Finance Program,' `Neurology Department,' `African Studies' or any other
sub-field or entity.''\autocite{emory-style}
\item \Verb|\advisor{Advisor Name}{Advisor Degree}| sets up the name and degree of your 
adivsor. ``If your committee had co-chairs, refer to each of them as Advisor,
repeat this command, and list them in alphabetical order.''\autocite{emory-style} 
\item \Verb|\committee{Committee Member Name}{Committee Member Degree}|  sets up
the name and degree of your committee member. Repeat this command for multiple 
members. If the number of committee members exceeds 4, the class will change
the signing sheet into a two column style instead of one column style.
\item \Verb|\yr{2016}| sets up the calendar year you receive your degree
(not the academic year).
\end{itemize}


\section{Recommended packages}
More external packages can be loaded via \Verb|\usepackage[options]{package}|
in the preamble. The use of packages is completely the choice of the author,
and here are a few commonly used \LaTeX{} packages recommended to new \LaTeX{}
users.

All the packages listed here are standard \LaTeX{} packages that should come
with any standard \LaTeX{} distribution. If you cannot find a certain package 
in your local computer system, check out the CTAN archive \url{https://www.ctan.org/}.

\begin{itemize}
\item \Verb|amsmath| provides AMS standard mathematical enviroments for equations and symbols.
\item \Verb|graphix| provides extended figure environment to treat various image formats.
\item \Verb|mhchem| provides easy typesetting of chemical formulae.
\item \Verb|longtable| provides extended table environment for extreme long tables
across several pages.
\item \Verb|multirow| ``creates tabular cells spanning multiple rows'' in the table environment.
\item \Verb|subfigure| provides extended figure environment for subfigure captions. 
\item \Verb|verbatim| provides extended verbatim environment that allows customized 
frames, position, and layout.
\item \Verb|url| generates hyperlinks for url in the final \Verb|pdf| output.
\end{itemize}


\section{Use environments}\label{sec:env}
The Emory class should be fully compatible with standard \LaTeX{} enviroments:
\Verb|figure|, \Verb|table|, \Verb|quote|, \Verb|itemize|, \Verb|enumerate|,
and \Verb|verbatim|.
%In addition to these standard enviroments, \Verb|emory.cls| also defines two new
%enviroments: \Verb|prose| and \Verb|poetry|, for bulk quotation of different literature genre.

Sample outputs of these enviroments are listed here. 

\subsection{Figures}

\begin{figure}[htp!]
  \centering
  \includegraphics[width=\textwidth]{LaTeXLogo.png}
  \caption{This is a figure enviroment.\label{fig}}
\end{figure}

\subsection{Schematics}

\Verb|schematic| is a customized environment inherieted from the \Verb|figure|
environment. It changes the caption head from ``Figure'' to ``Schematic'', and 
automatically centering the element. Do not include float position option
\Verb|[htp!]|, as it is already set up in the environment. It will still show
up in the list of figures, and will have consistent numbering with the \Verb|figure|
floats.

\begin{schematic}
  \includegraphics[width=\textwidth]{LaTeXLogo.png}
  \caption{This is actually not a schematic.\label{schematic}}
\end{schematic}


\subsection{Tables}

\begin{table}[htp!]
  \centering
  \caption{This is a simple table. \label{table}}
  \begin{tabular}{l l}
    \hline
    Sample & Table \\
    \hline 
    row & 1 \\
    row & 2 \\
    \hline
  \end{tabular}
\end{table}

\subsection{Mathematical equations}
Single mathematical equation (need \Verb|amsmath| package)
\begin{equation}
  E = mc^2 \label{math:single}
\end{equation}

Multiple mathematical subequations (need \Verb|amsmath| package)
\begin{subequations}\label{math:sub}
\begin{align}
  \nabla\cdot\pmb{E}&=\frac{\rho}{\epsilon_0} \label{math:sub1}\\
  \nabla\cdot\pmb{B}&=0 \label{math:sub2} \\
  \nabla\times\pmb{E}&=-\frac{\partial\pmb{B}}{\partial t} \label{math:sub3}\\
  \nabla\times\pmb{B}&=\mu_0\Big(\pmb{J}+\epsilon_0\frac{\partial\pmb{E}}{\partial t}\Big) \label{math:sub4}
\end{align}
\end{subequations}

\subsection{Block quotes}
Quote enviroment. Use \Verb|\begin{quote} \end{quote}|

\begin{quote}\label{poet}
O wild West Wind, thou breath of Autumn's being,

Thou, from whose unseen presence the leaves dead

Are driven, like ghosts from an enchanter fleeing,
\end{quote}

\subsection{Lists and enumerates}
List and enumerate environment. Use \Verb|\begin{itemize} \item bla \end{itemize}|
and \Verb|\begin{enumerate} \item bla \end{enumerate}|. List and enumerate environments
can be nested.

A nested list structure:
\begin{itemize}
  \item This is a list item.
  \begin{itemize}
    \item This is a nested list item.
    \begin{itemize}
      \item This is a nested nested list item.
      \begin{itemize}
        \item This is a nested nested nested list item. 
        \item The bullets for 4 levels are defined in \Verb|emory.cls|.
        \item I would strongly doubt you will need more than 4 levels of nested lists.
      \end{itemize}
    \end{itemize}
  \end{itemize}    
\end{itemize}

A nested enumerate structure:
\begin{enumerate}
  \item This is a enumerate item.
  \begin{enumerate}
    \item This is a nested enumerate item.
    \begin{enumerate}
      \item This is a nested nested enumerate item.
      \begin{enumerate}
        \item This is a nested nested nested enumerate item. 
        \item The numbering for 4 levels are defined in \Verb|emory.cls|.
        \item I would strongly doubt you will need more than 4 levels of nested lists.
      \end{enumerate}
    \end{enumerate}
  \end{enumerate}    
\end{enumerate}

\section{Verbatim}

Use \Verb|\verb?<code>?| to typeset inline code  (\Verb|?| can be replaced by
other paired special characters),
or  \Verb|\begin{verbatim} <code> \end{verbatim}| to typeset block codes.
Package \Verb|fancyvrb| offers inline \Verb|\Verb??| and \Verb|Verbatim| environment with more enhanced
options (notice the ``V'' is capitalized).

\begin{Verbatim}
# a block of code
print 'Hello world'
def foo(bar):
    renturn bar
\end{Verbatim}


Use \Verb|\usepackage[version=3]{mhchem}| for your chemicals, like this:
\Verb|\ce{C2H5OH}| makes \ce{C2H5OH}.


\section{Cross reference}
One advantage of \LaTeX{} out of many is the convinience of cross reference in
the text. Use \Verb|\label{yourlabel}| to label the material, and \Verb|\ref{yourlabel}|
to reference it in some other places in your text. And you do not need to worry
about numbering: \LaTeX{} does it automatically for you. For example, here
I will reference the sample enviroments in the previous section (Section \ref{sec:env}):
\begin{itemize}
\item The label for Table~\ref{table} is \Verb|\label{table}|.
\item The label for Figure~\ref{fig} is \Verb|\label{fig}|.
\item The label for Equation~\ref{math:single} is \Verb|\label{math:single}|
\item The label for Equations~\ref{math:sub} is \Verb|\label{math:sub}|. And then
      you can reference to each subequation: \ref{math:sub1}, \ref{math:sub2}, 
      \ref{math:sub3}, and \ref{math:sub4} via \Verb|\label{math:sub1}|,
      \Verb|\label{math:sub2}|, \Verb|\label{math:sub3}|, and \Verb|\label{math:sub4}|,
      respectively.
\item The label for Quote~\ref{poet} is \Verb|\label{poet}|.
\end{itemize}

\section{Citation and bibliography}
Another great advantage of \LaTeX{} is its bibliography manager: \BibTeX{}.

Include all of you citations in a \Verb|.bib| file, and specify it at the end
of the document using command \Verb|\bibliography{sample-bib}|, before \Verb|\end{document}|.
(Do not include the extension \Verb|.bib|).

Each bibliography item has a ``bibkey''. Cite this ``bibkey'' in your text
using \Verb|\autocite{bibkey}| results in a correctly labeled citation.

%Two styles are available in this class: the name-date style, or the numeric-label 
%style. The style can be specified in the preamble (see Section~\ref{sec:options}).
%The citation formats of different resource types are automatically set up by the 
%class.

The number of bibliography style combinations is almost infinite. A \BibTeX{} 
style file \Verb|.bst| specially designed for the Emory class will be ideal, 
but requires significant amount of effort. This specialized \BibTeX{}
style file that understands the user input class options is still under development. 
Currently, a work round is to include your own \BibTeX{} style files, mostly likely
standard style files from CTAN. 

Here are some examples of the numeric-label style.
\begin{itemize}
\item Cite an article \autocite{bib-article}
\item Cite a book \autocite{bib-book}
\item Cite a book chapter \autocite{bib-chapter}
\item Cite a conference proceeding \autocite{bib-conference}
\item Cite an online resource (url) \autocite{bib-url}
\item Cite a unpublished document \autocite{bib-unpublished}
\end{itemize}


After update the bibliography file, you need to re-compile your document in the 
following manner:
\begin{enumerate}
\item \Verb|pdflatex sample|  updates the tex body, and writes new citation labels
into the \Verb|.aux| file
\item \Verb|bibtex sample|  reads the information from the \Verb|.bib| file, 
and writes it into the \Verb|.aux| file with the correct citation format.
\item \Verb|pdflatex sample|  run this command again, which reads the bibliography
items in the \Verb|.aux| file, and insert them into the correct location in the 
output document.
\item \Verb|pdflatex sample| run this command once again, which updates the 
numbering of the citation markers. 
\end{enumerate}
Do not include the extension \Verb|.tex| in the command line.


\appendix
\renewcommand{\chaptername}{Appendix}

\chapter{Appendix1}
This is an appendix.

\chapter{Appendix2}
This is an another appendix.

\end{document}
