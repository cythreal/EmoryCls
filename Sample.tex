% ------------------------- emory.cls Sample Document --------------------------
% This is a sample document for emory.cls 
% This document includes the contents likely to be included in a dissertation,
% such as university required cover pages, table of contents, list of tables
% and figures, chapters, floats (figures, tables, math equations, etc.), 
% citations, footnotes, appendices, and indices. 
% Compile this document using the emory.cls class file should sucessfully 
% generate a .pdf file with the desired dissertation format.
%
% The dissertation style instructions can be found at
% ------ documentclass declaration ---------------------------------------------
% --- \documentclass[options]{emory}
% --- Only [draft] or [final] is allowed. Other properties can be set by keys.
\documentclass[draft]{emory}

% --- Set up other document properties. Available keys and values:
% --- style: select style by major=<major>
%            supported majors: chem, phys, math, bio, and cs 
% --- print: doubleside = <boolean>
%            true: double sided printing [default]
%            false: single sided printing
% --- doc: dissertation = <boolean>
%          true: document is a dissertation [default]
%          false: document is a thesis

\typeout{I'm loading keys}
\setkeys[emory]{style}{major=chem}
\setkeys[emory]{print}{doubleside=true}
\setkeys[emory]{doc}{dissertation=true}



% ------ include your own packages ---------------------------------------------
\usepackage{url}


% ------ title and author and the boss -----------------------------------------
\title{A \LaTeX{} Class for the Dissertation Style of Emory University}
\author{Luyao Zou}                      % replace with your legal name
\degree{Ph.D.}   % replace with your degree, if necessary
\prevdegree{B.S.}    % replace with your previous degree
\field{Your field of expertise}
% \advisor{Name}{Degree}, same for committee.
% Repeat command for multiple inputs. Please sort with alphabatical order
\advisor{Your dear advisor}{Ph.D}
\advisor{Your co-advisor}{Ph.D}
\committee{Dr.\ Alfalfa}{Ph.D}
\committee{Dr.\ Banana}{M.D}
\committee{Dr.\ Cashew}{M.D \& Ph.D}
\committee{Dr.\ Dodge}{J.D}
\committee{Mr.\ Eggplant}{M.S.}
\yr{2016}                               % replace the year, if necessary


% ------ abstract --------------------------------------------------------------
\abstract{%
This is a \LaTeX{} class. It is supposed to follow the dissertation format
requirement of Emory University, but nothing is guaranteed, and please tweak the
code when necessary. 
}

% ------ acknowledgment -------------------------------------------------------
\acknowledgment{%
I would like to thank all the documentations and anonymous people online that
helped me learn the skills to write this class. 
Special thank to the great example of 
%\begin{itemize}
%  \item The ACS package \verb|achemso|
%  \item The thesis class originally written by Bruce Fast, and most recently updated by Hongcheng Ni, from University of Colorado, Boulder. 
        %The webpage is at 
        %\url{
        %http://www.colorado.edu/oit/software-hardware/tex-latex/thesis-class/documentation-sample-files
        %}
  %\item \href{https://latex-project.org/guides/clsguide.pdf}{LaTeX2e for class and package writers}
  %\item \href{http://texdoc.net/texmf-dist/doc/latex/base/ifthen.pdf}{\verb|ifthen| package document}
%\end{itemize}
}

% ------ start your awesome writing here ---------------------------------------
\begin{document}

\chapter{Introduction}
%Putting up a dissertation following all the requirements or restrictions of
%your university policy may drive you nuts. This is why \LaTeX{} is glorious. 
%Unfortunately, Emory does not provide an ``official'' \LaTeX{} style package for
%her brilliant, hard-working Ph.D.\ candidates. I here try my best to offer a
%third-party \LaTeX{} class that hopefully covers most of the Emory policy, if not
%all.
This is a \LaTeX{} class designed for dissertation writing under the style and
format policy of the Laney Graduate School at Emory University. This class is
intended to reduce the effort for Emory graduate students in formating, thus 
to focus more on the writing. 

To report bugs and suggestions, please contact me on GitHub. To contribute, 
please fork my github repo at \url{https://github.com/luyaozou/EmoryCls.git}

\chapter{Usage}
\section{Load the class and compilation}
The Emory class can be loaded by declaring
\verb|\documentclass{emory}| in the \LaTeX{} preamble. The class file
\verb|emory.cls| needs to be in the same directory of the \LaTeX{} file to be 
compiled. 

Alternatively, you may copy \verb|emory.cls| into your local \TeX{} package 
directory, similar to \verb|C:\MiKTeX\tex\latex\emory\| in Windows systems, or\\
\verb|/usr/share/texmf-texlive/tex/latex/emory/| in Mac and Linux systems. After
creating an \verb|emory| folder and putting the class file in it, run
\verb|texhash| to update the \TeX{} package registery. Remember, you may need
root permission to complete these operations. 

Use \verb|pdflatex| instead of \verb|latex| as the compilation driver. This
is because certain packages, such as \verb|url|, relies on pdf driver. 


\section{Set document options}\label{sec:options}
The options of this class can be set in the \LaTeX{} preamble. They are 
self-explanotary in the \verb|sample.tex| file. 

\begin{itemize}
\item \verb|\title{Title}| sets up the title of the dissertation/thesis. The title 
will be reprinted several times in the special pages.
\item \verb|\author{Name}| sets up the name of the author. The name will be reprinted
several times in the special pages. 
\item \verb|\degree{Ph.D./Master}| sets up the degree you apply. It is either Doctor
of Philosophy or Master of Arts/Science/or another area.  There are a number of 
possible Master’s degree in some programs, particularly Music and Education.  
If you are unsure, consult with your Director of Graduate Studies.
\item \verb|\prevdegree{B.S.}| sets up your previous academic degree. List your 
degree, the name of the university, and the year you received the degree.  
For example, ``B.A., Yale University, 2004'' or ``M.Sc., University of Pennsylvania, 2003''.
You can list several degrees, including master’s degrees from your current 
program here at Emory.  Note that you can only list a master’s degree from your
Emory program if you actually applied for and received it; even if master’s 
degrees are given on the basis of candidacy, they are not automatic. 
\item \verb|\field{Field of Expertise}| sets up your field of study. Generally, 
your field is the subject-matter name of your program: English, Chemistry, 
Health Services Research and Health Policy, Educational Studies, Business, etc.  
The field does not include the word “program,” and does not include any sub-field
you specialize in.  For example, your field is not ``19th Century American Literature,''
``Finance Program,'' ``Neurology Department,'' ``African Studies'' or any other
sub-field or entity.
\item \verb|\advisor{Advisor Name}{Advisor Degree}| sets up the name and degree of your 
adivsor. If your committee had co-chairs, refer to each of them as Advisor,
repeat this command, and list them in alphabetical order. 
\item \verb|\committee{Committee Member Name}{Committee Member Degree}|  sets up
the name and degree of your committee member. Repeat this command for multiple 
members. If the number of committee members exceeds 4, the class will change
the signing sheet into a two column style instead of one column style.
\item \verb|\yr{2016}| sets up the calendar year you receive your degree
(not the academic year).
\end{itemize}


\section{Recommended packages}
More external packages can be loaded via \verb|\usepackage[options]{package}|
in the preamble. The use of packages is completely the choice of the author,
and here are a few commonly used \LaTeX{} packages recommended to new \LaTeX{}
users.

All the packages listed here are standard \LaTeX{} packages that should come
with any standard \LaTeX{} distribution. If you cannot find a certain package 
in your local computer system, check out the CTAN archive \url{https://www.ctan.org/}.

\begin{itemize}
\item \verb|amsmath| provides AMS standard mathematical enviroments for equations and symbols.
\item \verb|graphix| provides extended figure environment to treat various image formats.
\item \verb|mhchem| provides easy typesetting of chemical formulae.
\item \verb|longtable| provides extended table environment for extreme long tables
across several pages.
\item \verb|multirow| ``creates tabular cells spanning multiple rows'' in the table environment.
\item \verb|subfigure| provides extended figure environment for subfigure captions. 
\item \verb|verbatim| provides extended verbatim environment that allows customized 
frames, position, and layout.
\item \verb|url| generates hyperlinks for url in the final \verb|pdf| output.
\end{itemize}


\section{Use environments}\label{sec:env}
The Emory class should be fully compatible with standard \LaTeX{} enviroments:
\verb|figure|, \verb|table|, \verb|itemize|, \verb|enumerate|, and \verb|verbatim|.
In addition to these standard enviroments, \verb|emory.cls| also defines two new
enviroments: \verb|prose| and \verb|poetry|, for bulk quotation of different literature genre.

Sample outputs of these enviroments are listed here. 

\begin{figure}[htp!]
  \centering
  \includegraphics[width=\textwidth]{LaTeXLogo.png}
  \caption{This is a figure enviroment.\label{fig}}
\end{figure}

\begin{table}[htp!]
  \centering
  \caption{This is a simple table. \label{table}}
  \begin{tabular}{l l}
    \hline
    Sample & Table \\
    \hline 
    row & 1 \\
    row & 2 \\
    \hline
  \end{tabular}
\end{table}

Single mathematical equation (need \verb|amsmath| package)
\begin{equation}
  E = mc^2 \label{math:single}
\end{equation}

Multiple mathematical subequations (need \verb|amsmath| package)
\begin{subequations}\label{math:sub}
\begin{align}
  \nabla\cdot\pmb{E}&=\frac{\rho}{\epsilon_0} \label{math:sub1}\\
  \nabla\cdot\pmb{B}&=0 \label{math:sub2} \\
  \nabla\times\pmb{E}&=-\frac{\partial\pmb{B}}{\partial t} \label{math:sub3}\\
  \nabla\times\pmb{B}&=\mu_0\Big(\pmb{J}+\epsilon_0\frac{\partial\pmb{E}}{\partial t}\Big) \label{math:sub4}
\end{align}
\end{subequations}

Poetry enviroment. Use \verb|\begin{poetry}{title}{poet}{year}} \end{poetry}|

\begin{poetry}{Ode to the West Wind}{P.\ B.\ Shelley}{1792--1822}\label{poet}
O wild West Wind, thou breath of Autumn’s being,

Thou, from whose unseen presence the leaves dead

Are driven, like ghosts from an enchanter fleeing,
\end{poetry}


Use \verb|\usepackage[version=3]{mhchem}| for your chemicals, like this:
\verb|\ce{C2H5OH}| makes \ce{C2H5OH}.


\section{Cross reference}
One advantage of \LaTeX{} out of many is the convinience of cross reference in
the text. Use \verb|\label{yourlabel}| to label the material, and \verb|\ref{yourlabel}|
to reference it in some other places in your text. And you do not need to worry
about numbering: \LaTeX{} does it automatically for you. For example, here
I will reference the sample enviroments in the previous section (Section \ref{sec:env}):
\begin{itemize}
\item The label for Table~\ref{table} is \verb|\label{table}|.
\item The label for Figure~\ref{fig} is \verb|\label{fig}|.
\item The label for Equation~\ref{math:single} is \verb|\label{math:single}|
\item The label for Equations~\ref{math:sub} is \verb|\label{math:sub}|. And then
      you can reference to each subequation: \ref{math:sub1}, \ref{math:sub2}, 
      \ref{math:sub3}, and \ref{math:sub4} via \verb|\label{math:sub1}|,
      \verb|\label{math:sub2}|, \verb|\label{math:sub3}|, and \verb|\label{math:sub4}|,
      respectively.
\item The label for Quote~\ref{poet} is \verb|\label{poet}|.
\end{itemize}

\section{Citation and bibliography}
Another great advantage of \LaTeX{} is its bibliography manager: {\rm B\kern-.05em{\sc i\kern-.025em b}\kern-.08em T\kern-.1667em\lower.7ex\hbox{E}\kern-.125emX}

Include all of you citations in a \verb|.bib| file, and specify it at the end
of the document using command \verb|\bibliography{sample-bib}|, before \verb|\end{document}|.
(Do not include the extension \verb|.bib|).

Each bibliography item has a ``bibkey''. Cite this ``bibkey'' in your text
using \verb|\cite{bibkey}| results in a correctly labeled citation.

Two styles are available in this class: the name-date style, or the numeric-label 
style. The style can be specified in the preamble (see Section~\ref{sec:options}).
The citation formats of different resource types are automatically set up by the 
class.

Here are some examples of the numeric-label style.
\begin{itemize}
\item Cite an article \cite{bib-article}
\item Cite a book \cite{bib-book}
\item Cite a book chapter \cite{bib-chapter}
\item Cite a conference proceeding \cite{bib-conference}
\item Cite an online resource (url) \cite{bib-url}
\item Cite a unpublished document \cite{bib-unpublished}
\end{itemize}


After update the bibliography file, you need to re-compile your document in the 
following manner:
\begin{enumerate}
\item \verb|pdflatex sample|  updates the tex body, and writes new citation labels
into the \verb|.aux| file
\item \verb|bibtex sample|  reads the information from the \verb|.bib| file, 
and writes it into the \verb|.aux| file with the correct citation format.
\item \verb|pdflatex sample|  run this command again, which reads the bibliography
items in the \verb|.aux| file, and insert them into the correct location in the 
output document.
\item \verb|pdflatex sample| run this command once again, which updates the 
numbering of the citation markers. 
\end{enumerate}
Do not include the extension \verb|.tex| in the command line.

\chapter{Random}
\section{This is gonna be a brilliant dissertation!}
\subsection{Born}
My arrival to this planet is an entire oopsy. 


\subsection{This is why I am awesome.}
\subsubsection{No mumble jumble}
The reason why I am awesome is because I do not use \verb|lipsum|. I write 
serious stuff in this sample file.







\bibliography{SampleBib}

\end{document}
